\section{Related work}
Several experiments have been done with OPC-UA and IIoT-like devices like Raspberry Pi's. In \cite{opcUaArt} different scenarios and use cases of the OPC-UA standard has been simulated.
The conclusions from the work in \cite{opcUaArt} is that application based on the OPC-UA standard has another type of flexibility than application with only a database. Another conclusion from the same article is the challenges to choose a OPC-UA software that's suited for the different tested cases. In \cite{} the possibilities to use a REST-api together with OPC-UA is investigated and tested. In their conclusion both positive and negative aspects with using the REST-api is mentioned. Positive aspects is that it's well scaled since the server doesn't need to any session information at all. Packets being sent between the client and the server is another benfit in using the REST-api. But with the positive aspects also negative aspects come into play. In the research the security has not been an aspect at all.