\section{Övning ett - Icebreaker}


\subsection{Bakgrund}
Vid möte med en ny grupp människor är det i början ofta stelt mellan individerna då det är första gången man träffas. Det kan vara svårt att hitta något att prata om innan man känner varandra. För att lätta upp stämningen och få reda på information om gruppens medlemmar kan man utföra en så kallad icebreaker-övning. Denna övningsform \" tvingar \" medlemmarna att ställa frågor till varandra. Det här resulterar i att man lär känna varandra på ett avslappnat sätt. \cite{icebreaker}


\subsection{Genomförande}
I denna icebreaker-övning gjordes två olika övningar.

\subsubsection{Övning ett}
I den första övningen har ett antal förutbestämda frågor skrivits ned på lappar. Dessa lappar lades sedan ut på ett bord. Den person som börjar väljer en lapp från högen och läser upp denna högt inför gruppen. Personen har även fått en boll vilket denna ger till den gruppmedlem som personen vill ska svara på frågan. Vid överlämnandet av bollen nämnet även personen gruppmedlemens namn. Den tilldelade gruppmedlemen svarar på frågan och väljer därefter en egen lapp från högen och ger bollen vidare. Samtidigt nämner personen gruppmedlemens namn.


\bigskip

Detta tillvägagångssätt fortlöper tills frågorna på bordet är slut.

\subsubsection{Övning två}
I den andra övningen börjar en av gruppmedlemmarna att säga en sanning om sig själv. Detta kan till exempel vara \"Jag har en bror\". Om någon av gruppmedlemmarna stämmer in på denna sanning ställer krokar personerna arm med varandra. Den person som krokar arm med den första personen ställer nu en fråga till gruppen på samma samma sätt. Övningen avslutas när gruppmedlemmarna har skapat en cirkel med varandra.


\subsection{Resultat}
Resultatet från de två övningarna utföll bra. Gruppens medlemmar glömde ibland bort att säga den andra personens namn när bollen gavs över i den första övningen.

\bigskip

Den andra övningen utfördes totalt tre gånger där det i varje runda blev svårare att hitta ett påstående som stämde överens med en annan gruppmedlem. Det här då tidigare påståenden inte användes en andra gång.