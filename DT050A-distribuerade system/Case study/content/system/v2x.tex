\subsection{Vehicle-to-everything} \label{v2xSec}
The Vehicle-to-everything, or V2X, is a vehicular communication system that is used to handle communication between vehicles and all kind of devices.
The V2X has two different communication technologies. WLAN and Cellular. 

\subsubsection{WLAN-model} \label{v2iWlanMod}

The WLAN model was published by the IEEE in 2012 with the official name IEEE 802.11p. This model is using Dedicated Short Range Communication (DSRC) which is using the underlaying radio provided in the model. The model is operating in the 5.9 GHz frequenzy band which is used for short range communication and has low latency. \cite{dsrc}

\subsubsection{Cellular-model} \label{v2iCellMod}

The cellular based model has been developed in recent versions of the V2X model and is called C-V2X. It's operating in the LTE band and will support future generations of the mobile network protocol such as 5G. \cite{gsma}

\bigskip

Advantages over the WLAN-model is that the vehicle can communicate with more devices since it's working in a longer range than the IEEE 802.11p. It's also more secure and has higher speed to mention a few of the advantages. \cite{gsma}

\bigskip

The cellular model supports two types of communication;

\begin{enumerate}
    \item Direct communication
    \item Cellular based communication
\end{enumerate}

When the model is talking directly with other devices it works just like the WLAN model. When a link is established between two devices an interface called PC5 is used. The PC5 interface is establishing a link between the vehicle and the other device.
In the case when the vehicle is communicating through a base station the link is established via a Uu interface which makes up the logical interface.