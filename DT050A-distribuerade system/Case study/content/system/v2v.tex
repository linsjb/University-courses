\section{Vehicle-to-vehicle} \label{v2vSec}
The Vehicle-to-vehicle, or V2V, is a communication model is used for communication directly between two or more vehicles. Ususally this model is used to send speed and position between nerby vehicles. The communication is handled in one of two possible ways. Either via the WLAN-based model mention in \fullref{v2iWlanMod} or via direct communication with the cellular model, mention in \fullref{v2iCellMod}.

\bigskip

Since the vehicles is talking directly cars far apart from each other cannot talk to each other. To be able to send out and receive information on long distances the model is creating a mesh network between vehicles but also between road infrastructure such as road signs and light poles.

\bigskip

The goal with the model is to avoid accidents by give the vehicle possibility to assist the driver in dangerous situations where the driver might not be quick enough to react. A few scenarios of this are shown in \cref{fig:v2vSitu} below. It's also used to provide the driver with useful information about traffic situations in the immediate area. \cite{gsma}

\framedFig{H}{Different scenarios where V2V can improve traffic safety \cite{gsma}}{v2vSitu}{0.9}{fig/v2vsitu.png}