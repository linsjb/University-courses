\section{OPC-UA application information} \label{appImpl}
The experiment in this research paper has been formed as a software to evaluate how well low performance devices and the \acrshort{opcua} work together.


\subsection{Hardware} \label{appImplHardware}
The hardware used in the testbed is a \acrlong{rpi} 1 model B+ \cite{rpi1b} and \acrlong{rpi} 3 model B+ \cite{rpi3b}. Since \acrshort{opcua} does not specify any hardware specification both of these \acrshort{rpi} models has been used to extend the width of the test.
A Macbook pro 13 inch model \cite{mbp13} has been used as a reference system to simulate a powerful server machine.

The reference system has been included in the test to have something to measure the \acrshort{rpi} models against. All three machines has been running over 1 Gbit cable connected LAN Ethernet.


\subsection{Container based environment} \label{appImplSoftware}
Since the same tests been running on three different systems the container based environment Docker \cite{docker} has been used to mainstream the tests. This means that the software environment is equal for all three systems. 


\subsection{OPC-UA library} \label{appImplOpcuaLibrary}
The \acrshort{opcua} software used for both server and client is the free open source \acrshort{opcua} Python library \textit{opcua-asyncio} from \textit{Freeopcua} \cite{freeopcua}. The library is an asynchronous which enables concurrent running of multiply threadlike methods. This enables quicker programs since the program execution does not need to wait for each other.