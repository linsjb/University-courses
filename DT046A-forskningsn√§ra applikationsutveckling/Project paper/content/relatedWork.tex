\section{Related work} \label{relatedWork}
Several experiments has been made with the \acrshort{opcua} protocol and \acrshort{iiot}-like devices. In \cite{opcUaArt} the research team tested different types of simulated scenarios with the \acrshort{opcua}. In one of the research scenarios they test to build a quality detection system that was connected to a database. The \acrshort{opcua} server, which was a \acrfull{rpi}, then talk to the database to both send and retrieve various necessary data. The conclusion from this scenario it that a database connected \acrshort{opcua} application has another level of flexibility than a traditional database connected application.

Another scenario in \cite{opcUaArt} is a \textit{monitoring and control} system. The idea was to use \acrshort{opcua} to monitor a human and robot collaboration in a production cell. Here yet again a \acrshort{rpi} is used as the \acrshort{opcua} server. This server receives information form different web services around the production cell. To present information about the state of the cell they also developed a \acrshort{opcua} client.
The conclusion in this scenario shows that \acrshort{opcua} is made to be used with this type of systems where the server gather information from the, in this case, production cell. The data can then directly be realized in the \acrshort{opcua} server.