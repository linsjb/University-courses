\section{Övning två - Krav}

\subsection{Bakgrund}
I all typ av utveckling av en produkt är det viktigt att i ett första skede skapa en kravspecifikation för att veta hur produkten ska utformas. För att enklare kunna skapa dessa krav samt ha någon form av underlag kan man göra på ett par olika sätt. 

\subsubsection{Personas}
Personas bygger på att man skapar fiktiva personer med olika personligheter som kan tänkas använda den produkt man försöker utveckla. Utifrån dessa fiktiva personer kan man skapa användarmål samt problemområden med produkten som sedan ligger till grund för produktens funktionalitet.

\subsubsection{Användarhistorier}
Ett vidare steg i kravfångst är att skapa korta användarhistorier. Dessa användarhistorier kan på ett enkelt och tydligt sätt förklara de krav som ställs på produkten. Användarhistorier följer som oftast formatet [Roll] ska kunna [krav/funktion] för att [orsak].  \cite[s.110]{agilProj}.

\subsubsection{Kravprioritering}
Vid produktutvecklingen finns det oftast mer än en handfull krav för produkten. Vissa av dessa krav är viktigare än andra och det är således viktigt att man prioriterar de krav som finns för produkten. Vilka krav måste finnas med, vilka krav bör finnas och vilka krav kan vi ha med? Detta tillvägagångssätt kallas MoSCoW modellen.  Med denna modell kan man enkelt dela in kraven i olika prioriteringsområden. \cite[s. 118]{agilProj}

\newpage

\subsection{Genomförande}
Kravövningen genomfördes genom att gruppen tillsammans fick skapa ett antal personas. Utifrån dessa personas skapades en lista med användarmål samt en lista över problemområden. Dessa listor stod sedan som grund för att skapa funktioner för produkten med olika prioriteringar enligt MoSCoW-modellen som nämns ovan. De prioriteringskategorier som användes i övningen var;

\begin{itemize}
    \item Vill ha
    \item Bra att ha
    \item Måste ha
\end{itemize}

\Cref{fig:perFlow} nedan visar hur ''informationsflödet'' gick från personas till en prioriterad kravlista.

\fig{H}{''Informationsflöde'' från personas till prioriterad kravlista}{perFlow}{1.0}{personasFlow.png}

\bigskip

Vidare i övningen skapades korta användarhistorier för att förtydliga de krav som skapats med hjälp av personas. Som grund för historierna togs de krav som skapats i personas-övningen.


\subsection{Resultat}
Gruppen hade på förhand redan skapat en kravlista över var deras produkt behöver. Övningen genomfördes ändå för att ge gruppen en djupare förståelse för kravhantering samt stödjande underlag för deras krav.


\bigskip

I \cref{fig:figOne} presenteras resultatet från övningen första del där personas skapades. Tre personas skapades för att ge gruppen en förståelse i hur personas skapas och används.

\fig{H}{Skapade personas}{figOne}{1.0}{figOne.jpeg}

Utifrån personas skapades ett antal korta användarhistorier. Dessa presenteras i \cref{fig:figTwo}.

\fig{H}{Korta användarhistorier}{figTwo}{1.0}{figTwo.jpeg}

\subsection{Frågestund och avslut}
Efter övningen kunde gruppen ställa frågor om det vi gått igenom. Vi frågade även gruppen om det fanns några oklarheter i övningen. Gruppen förstod det vi gått igenom. Som avslut meddelade vi gruppen att de kan fortsätta jobba med sina krav på samma stätt som vi gått igenom ifall de kommer på fler personas och/eller användarhistorier.
