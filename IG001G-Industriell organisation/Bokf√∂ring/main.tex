\documentclass[12pt]{article}

\setlength{\parindent}{0em}

\usepackage[utf8]{inputenc}
\usepackage[swedish]{babel}
\usepackage[%
a4paper,
top=2.cm,
bottom=2cm,
left=3.7cm,
right=3.7cm
]{geometry}

\usepackage{minted}
\usepackage{listings}
\usepackage{graphicx}
\usepackage{cleveref}
\usepackage{csvsimple}

\usepackage[
backend=biber,
style=ieee,
sorting=none
]{biblatex}

\addbibresource{bibliography.bib}

\newcommand{\showNotes}{%
    \setbeameroption{show notes on second screen=right}
}

\newcommand{\notesInput}[1]{%
    \note{\input{\notesPath#1}}
}

\newcommand{\fig}[3]{%
    \figure
        \includegraphics[width=#1\textwidth]{#3}
        \caption{#2}
    \endfigure
}

\newcommand{\code}[3]{%
    \begin{listing}
        \caption{#1}
        \inputminted[frame=lines]{#2}{\codePath #3}
    \end{listing}
}

\newcommand{\tikzFig}[3]{%
    \figure
        \centering
        \resizebox{#1\textwidth}{!}{%
            \input{\tikzPath #3.tex}
        }%
        \caption{#2}
    \endfigure
}

\newcommand{\texTable}[2]{%
    \table
        \centering
        \caption{#1}
        \input{\tablePath #2}
    \endtable
}

\newcommand{\csvTable}[2]{%
    \table
        \centering
        \caption{#1}
        \csvautotabular{\tablePath #2}
    \endtable
}


\newenvironment{animatedItemize}{
    \begin{itemize}[<+- | alert@+>]}
    {\end{itemize}
}

\newenvironment{animatedEnumerate}{
    \begin{enumerate}[<+- | alert@+>]}
    {\end{enumerate}
}

\newenvironment{notedFrame}[2]{
    \begin{frame}{#1}
        \notesInput{#2}}
    {\end{frame}
}

\newenvironment{notedSection}[2]{
    \section{#1}
        \notesInput{#2}}
    {
}


\title{Frågor uppgift Bokföring \\ \large{IG001G - Introduktion till industriell ekonomi}}
\author{Linus Sjöbro}
\date{\today}

\begin{document}
\maketitle

\section*{Uppgift 1}
\subsection*{Fråga A}
Se excelblad \textbf{Verifikationer}

\subsection*{Fråga B}
Se excelblad \textbf{Bokföring} flik \textbf{Konton}

\subsection*{Fråga C}
Årets vinst blir $1 \: 188 \: 094 \: kr$

\subsection*{Fråga D}
Se excelblad \textbf{Bokföring} flik \textbf{Balansrapporter}

\section*{Uppgift 2}
Om datorn, som nu är fallet, inte överstiger prisbasbeloppet kan denna direkt betalas av. Så här behövs datorn bara bokföras in och inte avskrivas. Då jag valt fakturametoden har jag valt att dela in köpet i två verifikationer fast denna betalas av direkt. Kanske onödigt men jag anser att det blir renare på det här sättet då allt följer samma princip.

\bigskip

Om datorn däremot överstiger prisbasbeloppet måste denna istället avskrivas.
Antingen en linjär, degresiv eller progressiv avskrivning.

Den linjära avskrivningen följer samma belopp hela tiden. Datorn kostar $20000$ kr och låt säga att den ska betalas av på 5 år. Då blir alltså den årliga avskrivningen $\frac{100000}{5}=20000$ kr/år.

Vid degresiv avskrivning skrivs den av mest i början och minst i slutet. Låt säga $30 \%$ i början och $5 \%$ år 5.

progressiv avskrivning är tvärtemot degressiv där avskrivningen ökar för varje år.

\bigskip

Utelämnar produktionsberoende avskrivning då det inte är relevant.

\subsection*{Ändrade verifikat}
Dessa verifikat skulle behöva ändras i pågående räkenskapsår (förutsätt en avskrivningstid på 5 år. Alltså 20\% av värdet varje år);

\textbf{Ver. 10} 5411 ändras istället till ett tillgångskonto som 1221, 1250 eller dylikt. Vi tar även bort 2440 för att underlätta då det tekniskt sett inte behövs då köpet ändå utfördes direkt med betalkort.

\textbf{Ver. 11} Ändras helt. Konto 1229, Ack. avskrivning på inventarie/verktyg, krediteras med det årliga avskrivningsbeloppet ($\frac{20000}{5}= 4000$ kr)

Konto 7832, Avskrivning inventarie, debiteras med samma belopp.


\section*{Uppgift 3}
I kontantmetoden bokförs en kundfaktura eller leverantörsfaktura först när denna har betalats. Det finns alltså bara en verifikation för fakturan.

I fakturametoden bokförs fakturan direkt denna har inkommit eller skapats.

Jag har valt fakturametoden och vid användade av kontantmetoden kan alla verifikationer som går emot 2440, levarantörsskulder samt 1510, kundfordringar tas bort.




\end{document}