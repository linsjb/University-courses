\documentclass[12pt]{article}

\setlength{\parindent}{0em}

\usepackage[utf8]{inputenc}
\usepackage[swedish]{babel}
\usepackage[%
a4paper,
top=2.cm,
bottom=2cm,
left=3.7cm,
right=3.7cm
]{geometry}

\usepackage{minted}
\usepackage{listings}
\usepackage{graphicx}
\usepackage{cleveref}
\usepackage{csvsimple}

\usepackage[
backend=biber,
style=ieee,
sorting=none
]{biblatex}

\addbibresource{bibliography.bib}

\newcommand{\showNotes}{%
    \setbeameroption{show notes on second screen=right}
}

\newcommand{\notesInput}[1]{%
    \note{\input{\notesPath#1}}
}

\newcommand{\fig}[3]{%
    \figure
        \includegraphics[width=#1\textwidth]{#3}
        \caption{#2}
    \endfigure
}

\newcommand{\code}[3]{%
    \begin{listing}
        \caption{#1}
        \inputminted[frame=lines]{#2}{\codePath #3}
    \end{listing}
}

\newcommand{\tikzFig}[3]{%
    \figure
        \centering
        \resizebox{#1\textwidth}{!}{%
            \input{\tikzPath #3.tex}
        }%
        \caption{#2}
    \endfigure
}

\newcommand{\texTable}[2]{%
    \table
        \centering
        \caption{#1}
        \input{\tablePath #2}
    \endtable
}

\newcommand{\csvTable}[2]{%
    \table
        \centering
        \caption{#1}
        \csvautotabular{\tablePath #2}
    \endtable
}


\newenvironment{animatedItemize}{
    \begin{itemize}[<+- | alert@+>]}
    {\end{itemize}
}

\newenvironment{animatedEnumerate}{
    \begin{enumerate}[<+- | alert@+>]}
    {\end{enumerate}
}

\newenvironment{notedFrame}[2]{
    \begin{frame}{#1}
        \notesInput{#2}}
    {\end{frame}
}

\newenvironment{notedSection}[2]{
    \section{#1}
        \notesInput{#2}}
    {
}

\graphicspath{{./attachments/figures/images/}}
\newcommand{\tikzPath}{./attachments/figures/tikz/}
\newcommand{\codePath}{./attachments/code/}
\newcommand{\tablePath}{./attachments/tables/}

\title{Kostnads- och investeringskalkyler \\ \large{Hemuppgift 1 - IG001G}}
\author{Linus Sjöbro}
\date{\today}

\begin{document}
\maketitle
\section{Självkostnadskalkyl}
\csvTable{H}{Självkostnadskalkyl}{}{selfCost/selfCostTable.csv}

\section{ABC-kalkyl}
\csvTable{H}{ABC - Parametrar}{}{abc/parametrar.csv}
\csvTable{H}{ABC - Aktiviteter}{}{abc/aktiviteter.csv}
% \csvTable{H}{ABC - Kostnadsdrivare}{}{abc/kostnadsdrivare.csv}
\csvTable{H}{ABC - Kalkyl}{}{abc/kalkyl.csv}

\section{Nuvärdeskalkyl}
Kassaflöde från uppgift enligt \cref{tab:cashFlow} nedan.

\csvTable{H}{Kassaflöde från uppgift}{cashFlow}{now/cashFlow.csv}

\subsection{Uppgift A - WACC}
WACC beräkning enligt \cref{eqa:wacc} nedan.

\begin{equation} \label{eqa:wacc}
p=\frac{K_e}{K}r_e + \frac{K_f}{K}(1-s)r_f
\end{equation}

\bigskip

Variabelbeskrivning:
\begin{enumerate}
    \item $P$ = WACC
    \item $K_e$ = Eget kapital
    \item $K_f$ = Skulder
    \item $K$ = Totalt kapital
    \item $r_e$ = Avkastningskrav eget kapital
    \item $r_f$ = Skuldränta
    \item $s$ = Bolagets skattesats
\end{enumerate}

Kapitalkällor från uppgiften i \cref{tab:kapKa} nedan.

%Kapitalkällor
\csvTable{H}{Kapitalkällor}{kapKa}{now/cap.csv}


WWAC beräkningar enligt \cref{tab:wwac} nedan.
%WWAC
\csvTable{H}{WWAC beräkningar}{wwac}{now/wwac.csv}

\subsection{Uppgift B - Nunettovärdesmetoden}

Nunettovärdesmetoden beräknas med hjälp av \cref{eqa:nnuv} nedan.

\begin{equation} \label{eqa:nnuv}
NNUV = \sum^n_{i=1}\frac{x_i}{(1+r)^n}-G
\end{equation}

\bigskip


Variabelbeskrivning:
\begin{enumerate}
    \item $n$ = Ekonomisk livslängd
    \item $r$ = Kalkylränta
    \item $x_i$ = Kassaflöde
    \item $G$ = Grundinvestering
\end{enumerate}

NV och NNUV enligt \cref{tab:nvCalc} nedan.

%NV beräkningar
\csvTable{H}{NV-beräkningar/produkt och år}{nvCalc}{now/nnuv.csv}


Den bäst lämpade produkten är \textbf{Produkt E} då denna har högst värde. Sammanställt resultat i \cref{tab:nnuv} nedan. 


\subsection{Uppgift C - Internränta}
Internräntan beräknas precis som NNUV men istället för kalkylräntan används istället den framtagna internräntan för att få NV att komma så nära 0 som möjligt.


%Beräkningar interränta
\csvTable{H}{Beräkning av internränta/produkt}{intern}{now/intern.csv}


Resultat återfinns i \cref{tab:nnuv}.
\subsection{Uppgift D - Annuitetsfaktor}

\csvTable{H}{Annuitetsberäkningar}{annu}{now/annu.csv}

Resultat återfinns i \cref{tab:nnuv}.

\subsection{Sammanställt resultat}
Sammanställning av samtliga beräkningar med efterfrågat svar.

NNUV beräknas enligt \cref{tab:nvCalc}.

Internräntan beräknas enligt \cref{tab:intern}.

Annuitetsvärde beräknas enligt \cref{tab:annu}.
%nunettovärdesmetoden
\csvTable{H}{Resultat för samtliga beräkningar förutom WWAC}{nnuv}{now/result.csv}





\end{document}